%%%%%%%%%%%%%%%%%%%%%%%%%%%%%%%%%%%%%%%%%%%%%%
%                insertmeeting
% 1) Title (something creative & funny?)
% 2) Date (MM/DD/YYYY)
% 3) Location (ex. Hagerty High School)
% 4) People/Committees Present 
% 5) Picture 
% 6) Start Time & Stop Time (ex. 12:30AM to 4:30PM)
%%%%%%%%%%%%%%%%%%%%%%%%%%%%%%%%%%%%%%%%%%%%%%
\insertmeeting 
{Mangekyo Sharingan!} 
{01/02/23} 
{Hagerty High School}
{Anouska, Mohana, Ritam, Samantha}
{Images/RobotPics/robot.jpg}
{2:30 - 5:30}

\hhscommittee{Software}
\noindent\hfil\rule{\textwidth}{.4pt}\hfil
\subsubsection*{Goals}
\begin{itemize}
    \item Program robot to drive toward cones

\end{itemize} 

\noindent\hfil\rule{\textwidth}{.4pt}\hfil

\subsubsection*{Accomplishments}
During autonmous period, our goal is to pick up a cone from the cone stack at the edge of the field and rop it on the high cone. To do this, we must first detect where the cone is, and the height of the cone stack to change the height of the elevator on our intake. To do detect the cone, we planned on using the OpenCV library to detect the Hue Saturation, and Value of the Blue and red cone. When we set this threshhold, we erode, and dialte to get the boundaries of the cone stack. We draw contours around the stack and get the hieght of the contour. This height tells us the height to move our cone intake to pick up the cone and be able to drive to the cone stack easier. This worked really well and we decided to keep working on it to integrate it into our Autonomous. 

\whatsnext{
\begin{itemize}
    \item Inegrate our cone vision program into autnomous and psossibly even teleOp
\end{itemize} 
}