%%%%%%%%%%%%%%%%%%%%%%%%%%%%%%%%%%%%%%%%%%%%%%
%                insertmeeting
% 1) Title (something creative & funny?)
% 2) Date (MM/DD/YYYY)
% 3) Location (ex. Hagerty High School)
% 4) People/Committees Present 
% 5) Picture 
% 6) Start Time & Stop Time (ex. 12:30AM to 4:30PM)
%%%%%%%%%%%%%%%%%%%%%%%%%%%%%%%%%%%%%%%%%%%%%%
\insertmeeting 
	{Never Going to Give You Up} 
	{01/13/22} 
	{Hagerty High School}
	{Annika, Anouska, Clayton, Falon, James, Jensen, Nathan, Ritam, Rose, Samantha, Lilly}
	{Images/RobotPics/robot.jpg}
	{2:30 - 4:30}
	
\hhscommittee{Software}
\noindent\hfil\rule{\textwidth}{.4pt}\hfil
\subsubsection*{Goals}
\begin{itemize}
    \item Implement our Road Runner classes into our robot's existing even-driven system. 

\end{itemize} 

\noindent\hfil\rule{\textwidth}{.4pt}\hfil

\subsubsection*{Accomplishments}
For today's extra long meeting, our goal was to create a class titled "TricycleRRDrive" that would mesh into our event-driven framework and the new localization methods we coded. The kinematic equations and localization processes are detailed in notebook entries over the last few days. 
To do this, we created a big method called "update" in the TricycleRRDrive class. This method will run every cycle and will handle our main logic. We created three states - IDLE, TURN, and FOLLOW TRAJECTORY that will each be called based on the robot's current task. TURN calculates the angle our robot should turn then uses the DriveSignal function we fixed a while ago to send power to the motors. Follow Trajectory takes the trajectory that we constructed and runs the robot along the path. Like before, it uses the DriveSignal function we created earlier. You'll also see some calls to the FtcDashboard. We've been running into some problems when combining Dashboard with our newer camera code, so the code doesn't work. However, we're confident that troubleshooting the errors caused by the camera will allow the dashboard to draw out our road runner paths. 
After that, we had to add the helper functions. The most important one is "driveSpline", allowing us to input a field coordinate and have the robot drive to it, automatically calculating road runner splines. It takes into account the direction we want the robot to travel and any constraints for the velocity. After we build the trajectory, the robot should go into the FOLLOW TRAJECTORY class. We simply send the power to the our existing drivebase class to make the robot move. Everything else we created (Kinematics, localization, and odometry) are working behind the scenes to keep everything updated. This feeds Road Runner accurate information to use. 
Thanks to Mr. Harper's help, we got the class running consistently. This project can be considered "complete" allowing us to move on to improving other parts of the robot. 
 
\begin{figure}[ht]
\centering
\begin{minipage}[b]{.48\textwidth}
  \centering
  \includegraphics[width=0.95\textwidth]{Meetings/January/01-13-22/1-13-22 pic1 - James Hu.JPG}
  \caption{The TURN state}
  \label{fig:011322_1}
\end{minipage}%
\hfill%
\begin{minipage}[b]{.48\textwidth}
  \centering
  \includegraphics[width=0.95\textwidth]{Meetings/January/01-13-22/1-13-22 pic2 - James Hu.JPG}
  \caption{The TRAJECTORY state}
  \label{fig:011322_2}
\end{minipage}
\end{figure}

\begin{figure}[ht]
\centering
\begin{minipage}[b]{.48\textwidth}
  \centering
  \includegraphics[width=0.95\textwidth]{Meetings/January/01-13-22/1-13-22 pic3 - James Hu.JPG}
  \caption{Our specific RRTricycleDrive extends the basic TricycleDrive class we created}
  \label{fig:011322_3}
\end{minipage}%
\hfill%
\begin{minipage}[b]{.48\textwidth}
  \centering
  \includegraphics[width=0.95\textwidth]{Meetings/January/01-13-22/1-13-22 pic4 - James Hu.JPG}
  \caption{Our method to turn the robot using kinematics}
  \label{fig:011322_4}
\end{minipage}
\end{figure}


\begin{figure}[htp]
\centering
\includegraphics[width=0.95\textwidth, angle=0]{Meetings/January/01-13-22/1-13-22 pic5 - James Hu.JPG}
\caption{The way we connect our calculations class (RRTricycleDrive) to actual movement (HhsTricycleDrivebase)}
\label{fig:011322_5}
\end{figure}

\begin{figure}[htp]
\centering
\includegraphics[width=0.95\textwidth, angle=0]{Meetings/January/01-13-22/1-13-22 pic6 - James Hu.JPG}
\caption{Our important method to build and follow spline curves}
\label{fig:011322_6}
\end{figure}

  
\hhscommittee{Multimedia}
\noindent\hfil\rule{\textwidth}{.4pt}\hfil
\subsubsection*{Goals}
\begin{itemize}
    \item Now that our visual storyboard is complete, our task for this meeting is to begin drawing backgrounds and search for voice-over candidates.

\end{itemize} 

\noindent\hfil\rule{\textwidth}{.4pt}\hfil

\subsubsection*{Accomplishments}We took some inspiration from last year's award-winning Promote video and chose to keep the bright and colorful theme. The color palette combines the "Pastel" and "Light" color options in Photoshop. We also used darkened versions of these colors to use as line art. Our goal when drawing backgrounds was to incorporate as little line art as possible. We did this so the viewer would be more focused on the characters in each scene. We accomplished this clean look by sketching out the rooms, reducing the opacity on the layer, and then using the Marquee and brush tools to block in colors. A new aspect of this Promote video was the greater use of contrasting colors. Many members of our committee are artists, and we applied our knowledge of contrasting colors to create the backgrounds. For example, in the first and last scenes in the Little Fire Guy's basement, the walls, floor, and furnishings are violet, but there are extra details on the shelves in warm colors like red, orange, and yellow. We love this scene in particular because the colors are pretty and set the standard of quality high for the rest of the video. We completed our meeting by finishing the 3 frames that would be used at the beginning and end of the video, and have begun working on an ending card to display the members behind the video and the music we used.

\begin{figure}[ht]
\centering
\begin{minipage}[b]{.48\textwidth}
  \centering
  \includegraphics[width=0.95\textwidth]{Meetings/January/01-13-22/1_13_22 - Falon Jones.png}
  \caption{Screenshots from our Promote video}
  \label{fig:011322_7}
\end{minipage}%
\hfill%
\begin{minipage}[b]{.48\textwidth}
  \centering
  \includegraphics[width=0.95\textwidth]{Meetings/January/01-13-22/1.13.22 - Falon Jones.png}
  \caption{Promote video screenshots}
  \label{fig:011322_8}
\end{minipage}
\end{figure}


\begin{figure}[htp]
\centering
\includegraphics[width=0.95\textwidth, angle=0]{Meetings/January/01-13-22/1_13_22 - Falon Jones (1).png}
\caption{More Promote screenshots}
\label{fig:011322_9}
\end{figure}

\whatsnext{
\begin{itemize}
    \item Finish commenting the class
	\item Move on to more camera work and fixing autonomous
    \item Our goal for the next meeting is to continue working on the background images for the frames, and continue drawing the ending card.

\end{itemize} 
}

