%%%%%%%%%%%%%%%%%%%%%%%%%%%%%%%%%%%%%%%%%%%%%%
%                insertmeeting
% 1) Title (something creative & funny?)
% 2) Date (MM/DD/YYYY)
% 3) Location (ex. Hagerty High School)
% 4) People/Committees Present 
% 5) Picture 
% 6) Start Time & Stop Time (ex. 12:30AM to 4:30PM)
%%%%%%%%%%%%%%%%%%%%%%%%%%%%%%%%%%%%%%%%%%%%%%
\insertmeeting 
	{Promote Party} 
	{11/16/21}
	{Hagerty High School}
	{Annika, Falon, Rose, Samantha}
	{Images/RobotPics/robot.jpg}
	{2:30 - 4:30}
	
\hhscommittee{Multimedia}
\noindent\hfil\rule{\textwidth}{.4pt}\hfil
\subsubsection*{Goals}
\begin{itemize}
    \item During this meeting, we should begin to plan out our ideas for the Promote Video. This includes reading the rubric and writing down our interpretations of this season's prompt. 

\end{itemize} 

\noindent\hfil\rule{\textwidth}{.4pt}\hfil

\subsubsection*{Accomplishments}
After reading through the rubric in the Game Manual, we verbally discussed our personal connections to the prompt. We also established a rough schedule for deadlines. For example, we want to have our brainstorming and script completed by winter break, and work on assembling the animation elements throughout January. This way, we will have plenty of time to put our full effort into each element of the video-making process, and complete it in a timely manner.


\whatsnext{
\begin{itemize}
    \item Next meeting, we are going to work on coming up with a rough draft of a storyboard for our video. We will begin by writing out main ideas for what we want to include in the video, and finding ways to connect them within the time constraints.
\end{itemize} 
}

