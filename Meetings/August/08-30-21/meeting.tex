% 1) Title
% 2) Date
% 3) Location
% 4) Present
% 5) Picture
% 6) Start Time
% 7) Stop Time
\insertmeeting 
	{FLL Figures} 
	{08/30/21}
	{Hagerty High School}
	{Annika, Clayton, Falon, Jensen, Nathan, Rose, Samantha}
	{Images/RobotPics/robot.jpg}
	{2:30}
  	{4:30}
	
\section*{Outreach}
\noindent\hfil\rule{\textwidth}{.4pt}\hfil
\subsection*{Goals}
\begin{itemize}
    \item Discuss plans for FLL teams
\end{itemize} 

\noindent\hfil\rule{\textwidth}{.4pt}\hfil

\subsection*{Accomplishments}
During today’s meeting we discussed our plans for our FLL teams this year. As one of our largest outreach events our FLL teams are extremely important to us. Because they are part of the hagerty robotics junior program, our teams are mentored entirely by 4717 members. Although the junior program is a key part of our sustainability plan, our teams are much more than a recruitment opportunity. Our main goal with our FLL teams is to spread our passion for stem and FIRST and create a community that shows kids not only how to become good engineers, but build their teamwork and emphasize gracious professionalism. As we talked about in our meeting, the benefit of having 4717 members as mentors is to be able to connect better to the kids than if there were adult members as well as to bring our experiences from FTC that will help us more effectively advise them.
Although we have done FLL in the past few years, we want to grow all of our teams and need to change some things from last year. Because we are back in person, we are able to structure the meetings differently and give the challenge students more independence than we could last year. With our first meeting being only a couple weeks away, we started planning, starting with how we would divide the teams. We plan on dividing the students into 1 Discover group, 3 Explore teams, and 2 Challenge teams. For challenge, we wanted to divide the kids into a younger, less experienced team and an older, more experienced team. Our intent in doing this was that we could spend more time teaching the less experienced team while the more experienced members wouldn't have to be taught things they already know. This will keep the program fun while still allowing the kids to learn and grow as a team. We agreed that for the first couple of weeks, we would spend some of the time in meetings teaching the new members basic skills, while continuing to build the more experienced members’ skills by showing them more advanced concepts.
