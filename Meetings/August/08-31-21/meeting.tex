% 1) Title
% 2) Date
% 3) Location
% 4) Present
% 5) Picture
% 6) Start Time
% 7) Stop Time
\insertmeeting 
	{Summer Camp Craze!} 
	{08/31/21}
	{Hagerty High School}
	{Annika, Jensen, Nathan, Ritam, Samantha}
	{Images/RobotPics/robot.jpg}
	{9:00}
  {2:00}
	
\section*{Outreach}
\noindent\hfil\rule{\textwidth}{.4pt}\hfil
\subsection*{Goals}
\begin{itemize}
    \item Introduce potential incoming members to the hagerty robotics program
    \item Build Potential members' interest in robotics

\end{itemize} 

\noindent\hfil\rule{\textwidth}{.4pt}\hfil

\subsection*{Accomplishments}
As a program, Hagerty robotics planned and held a 4 day summer camp to help bring aspiring members into our program and teach them about FIRST and VEX robotics. Before the recruitment camp started, we came up with an overall plan, including several presentations about hardware, software, outreach, notebook, and multimedia. In addition to the presentations, we decided the best way for the campers to experience what robotics has to offer was to hold a mini competition based on summer camps we have held in the past. This competition had the campers build a robot, create an autonomous program and write an engineering notebook with help from our experienced team members. When the camp started, we divided the 11 campers into 4 smaller teams; 2 FTC and 2 VEX. Inside these teams, the campers got to experience what it is like to be part of the Hagerty robotics program and were able to gain experience and knowledge from the current members of the program. At the end of the camp, we held robot matches based on real FTC and VEX meets, utilizing an original minigame, named “purpole”, that the campers had spent the previous 3 days of the camp preparing for. Overall, the camp was very successful in building the campers’ interest in the FIRST and VEX competitions, and many of them signed up to become members of the Hagerty robotics program in the upcoming season.