%%%%%%%%%%%%%%%%%%%%%%%%%%%%%%%%%%%%%%%%%%%%%%
%                insertmeeting
% 1) Title (something creative & funny?)
% 2) Date (MM/DD/YYYY)
% 3) Location (ex. Hagerty High School)
% 4) People/Committees Present 
% 5) Picture 
% 6) Start Time & Stop Time (ex. 12:30AM to 4:30PM)
%%%%%%%%%%%%%%%%%%%%%%%%%%%%%%%%%%%%%%%%%%%%%%
\insertmeeting 
	{Approaching Apexes} 
	{02/15/22} 
	{Hagerty High School}
	{James, Jensen, Samantha, Anouska, Annika, Clayton, Falon, Nathan, Ritam}
	{Images/RobotPics/robot.jpg}
	{2:30 - 4:30}
	
\hhscommittee{Software}
\noindent\hfil\rule{\textwidth}{.4pt}\hfil
\subsubsection*{Goals}
\begin{itemize}
    \item Continue refining autonomous programs. 

\end{itemize} 

\noindent\hfil\rule{\textwidth}{.4pt}\hfil

\subsubsection*{Accomplishments}
Today we were focused on fixing small problems with some of our autnomous programs. The first problem that we tackled today was inconsistency in a drive path. For the autonomous far near the carousel, we told the robot to splineTo a specific point in between the barrier and the wall to enter the warehouse. Our measurements seemed to be a bit off, so we remeasured the point with a tape measure.
The second thing we fixed today was a small bug where the arm was moving when it supposed to. Although we checked our autonomous program throughly, there were no instances where we set the arm power. After logging some debugging information and running states individually, we found that one of our movement methods was accidently calling an arm movement. 


\whatsnext{
\begin{itemize}
    \item Continue improving autonomous
\end{itemize} 
}

