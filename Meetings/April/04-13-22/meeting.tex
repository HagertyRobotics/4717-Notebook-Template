%%%%%%%%%%%%%%%%%%%%%%%%%%%%%%%%%%%%%%%%%%%%%%
%                insertmeeting
% 1) Title (something creative & funny?)
% 2) Date (MM/DD/YYYY)
% 3) Location (ex. Hagerty High School)
% 4) People/Committees Present 
% 5) Picture 
% 6) Start Time & Stop Time (ex. 12:30AM to 4:30PM)
%%%%%%%%%%%%%%%%%%%%%%%%%%%%%%%%%%%%%%%%%%%%%%
\insertmeeting 
	{Daredevil Drivers} 
	{04/13/22} 
	{Hagerty High School}
	{James, Jensen, Nathan, Ritam}
	{Images/RobotPics/robot.jpg}
	{2:30 - 4:30}
	
\hhscommittee{CAD}
\noindent\hfil\rule{\textwidth}{.4pt}\hfil
\subsubsection*{Goals}
\begin{itemize}
    \item Go over presentation and do driver practice

\end{itemize} 

\noindent\hfil\rule{\textwidth}{.4pt}\hfil

\subsubsection*{Accomplishments}
Today, Nathan, Ritam and I did driver practice for the first hour of the meeting. Then we as a team came together to practice the script and ensure that everything flowed correctly. From that practice, we determined that my part within the script, discussing materials and lightness of our robot, needed to be trimmed down in order to make everything else flow better and to make time for more important aspects of outreach. After that, Ritam worked on correcting Auto and I worked on ordering and drawing out the new designs for the new team element. The drawing itself is just supposed to get the idea across for the element. I plan on CADing the element tomorrow and printing most likely over the weekend. The entire concept of the new element is that there will be magnets inside the element that will essentially lock the flaps at 90 degree intervals. This will allow for less sagging and more stability on the element.


\whatsnext{
\begin{itemize}
    \item CADing and Printing the New element
\end{itemize} 
}
	
