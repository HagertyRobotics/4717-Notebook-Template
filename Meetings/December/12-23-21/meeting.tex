%%%%%%%%%%%%%%%%%%%%%%%%%%%%%%%%%%%%%%%%%%%%%%
%                insertmeeting
% 1) Title (something creative & funny?)
% 2) Date (MM/DD/YYYY)
% 3) Location (ex. Hagerty High School)
% 4) People/Committees Present 
% 5) Picture 
% 6) Start Time & Stop Time (ex. 12:30AM to 4:30PM)
%%%%%%%%%%%%%%%%%%%%%%%%%%%%%%%%%%%%%%%%%%%%%%
\insertmeeting 
	{We're Gaining Traction} 
	{12/23/21} 
	{Hagerty High School}
	{Anouska, James, Ritam, Samantha}
	{Images/RobotPics/robot.jpg}
	{2:30 - 4:30}
	
\hhscommittee{Software}
\noindent\hfil\rule{\textwidth}{.4pt}\hfil
\subsubsection*{Goals}
\begin{itemize}
    \item Add sliders to Eclipse so we can test different values  

\end{itemize} 

\noindent\hfil\rule{\textwidth}{.4pt}\hfil

\subsubsection*{Accomplishments}
We started off by adding sliders into Eclipse to continue our progress on the hough circles. We include the minimum, maximum, and starting value for the slider. We upload a photo of the circles on the team element rather than keeping the webcam on. This allowed us to see the changes made in the photo as we edit the values. Once we found the correct arguments for the maximum radius, minimum radius, dp, and minimum distance, we found that we could use just the dp value and minimum distance value. The dp parameter holds the ratio of the inverse ratio of the accumulator resolution to the image resolution. We made sure to find just two circles by seeing if the two circles found had the same x value. 
 



\whatsnext{
\begin{itemize}
    \item Move it from Eclipse to Android Studio
\end{itemize} 
}

