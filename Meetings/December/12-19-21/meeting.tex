%%%%%%%%%%%%%%%%%%%%%%%%%%%%%%%%%%%%%%%%%%%%%%
%                insertmeeting
% 1) Title (something creative & funny?)
% 2) Date (MM/DD/YYYY)
% 3) Location (ex. Hagerty High School)
% 4) People/Committees Present 
% 5) Picture 
% 6) Start Time & Stop Time (ex. 12:30AM to 4:30PM)
%%%%%%%%%%%%%%%%%%%%%%%%%%%%%%%%%%%%%%%%%%%%%%
\insertmeeting 
	{Meeting Example} 
	{12/19/21} 
	{Hagerty High School}
	{Ritam}
	{Images/RobotPics/robot.jpg}
	{10:00 - 5:00}
	
\hhscommittee{General}
\noindent\hfil\rule{\textwidth}{.4pt}\hfil
\subsubsection*{Goals}
\begin{itemize}
    \item Start editing roadrunner to our needs 

\end{itemize} 

\noindent\hfil\rule{\textwidth}{.4pt}\hfil

\subsubsection*{Accomplishments}
After a successful second meet, we realized that many other teams were starting to do the same tasks that we sought to complete in autonomous. This led to us thinking about how we could be different and work together with our alliance partner in autonomous, rather than us just sitting there or doing our normal autonomous routing of dropping the pre-loaded block, spinning the carousel and parking. We found that most of the time, the people who did the same tasks as us started on the side closest to the carousel as it was easier to do all of those tasks when close to the carousel. We found that if we started on the other side and picked up blocks, the 2 teams together could have a very high scoring autonomous, however our current way of writing our software with a basic PID controller would not suffice. This led us back to roadrunner, a motion planning library that allows for complex path following. This would be essential for us as we would need to perfectly be able to get through the gap autonomously and would allow us to be faster than just using the basic PID controller. We reverted back to when we had the roadrunner library in our directory and started to think about ways to change their programs up to work with our tricycle drive, however we first needed to figure out how everything worked. 
 

\whatsnext{
\begin{itemize}
    \item Figuring out how everything works together in roadrunner
\end{itemize} 
}

