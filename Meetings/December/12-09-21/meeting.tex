%%%%%%%%%%%%%%%%%%%%%%%%%%%%%%%%%%%%%%%%%%%%%%
%                insertmeeting
% 1) Title (something creative & funny?)
% 2) Date (MM/DD/YYYY)
% 3) Location (ex. Hagerty High School)
% 4) People/Committees Present 
% 5) Picture 
% 6) Start Time & Stop Time (ex. 12:30AM to 4:30PM)
%%%%%%%%%%%%%%%%%%%%%%%%%%%%%%%%%%%%%%%%%%%%%%
\insertmeeting 
	{Ready, Set, Action!} 
	{12/09/21} 
	{Hagerty High School}
	{Annika, Falon, Rose}
	{Images/RobotPics/robot.jpg}
	{2:30 - 4:30}
	
\hhscommittee{Multimedia}
\noindent\hfil\rule{\textwidth}{.4pt}\hfil
\subsubsection*{Goals}
\begin{itemize}
    \item Start script and storyboard  

\end{itemize} 

\noindent\hfil\rule{\textwidth}{.4pt}\hfil

\subsubsection*{Accomplishments}
Today we began the script and storyboard. We came up with a Little Fire Guy finding a letter written by an Old Fire Guy who grew up with FIRST. The letter is directed towards a younger self to show all the things Old Fire Guy learned while being a part of FIRST. We thought it would be interesting to make a video loop at the end by showing Old Fire Guy placing the letter in a box at the end and Little Fire Guy picking it up in the beginning. To show a progression in time, we use different color palettes for each scene. Each scene is a different part of Old Fire Guy's life, such as being a part of FLL, FTC, and then having a job. We wrote explanations below each scene describing the scene. 
When writing the script, we wrote out the key ideas we wanted to show in each scene. We decided on using a catchphrase "Take the wheel" as our overall answer to the prompt to connect it to this year's theme of transportation. We wrote out a draft of our script and used a website to convert it into the average words per minute it would take to read. We separated our script into the scenes it would be read in after editing and revising it. 
 
\whatsnext{
\begin{itemize}
    \item Begin drawing scenes
\end{itemize} 
}

