%%%%%%%%%%%%%%%%%%%%%%%%%%%%%%%%%%%%%%%%%%%%%%
%                insertmeeting
% 1) Title (something creative & funny?)
% 2) Date (MM/DD/YYYY)
% 3) Location (ex. Hagerty High School)
% 4) People/Committees Present 
% 5) Picture 
% 6) Start Time & Stop Time (ex. 12:30AM to 4:30PM)
%%%%%%%%%%%%%%%%%%%%%%%%%%%%%%%%%%%%%%%%%%%%%%
\insertmeeting 
	{Pattern Recognition Review} 
	{12/14/21} 
	{Hagerty High School}
	{Anouska, James, Ritam, Samantha}
	{Images/RobotPics/robot.jpg}
	{2:30 - 4:30}
	
\hhscommittee{Software}
\noindent\hfil\rule{\textwidth}{.4pt}\hfil
\subsubsection*{Goals}
\begin{itemize}
    \item Print new circles and continue to work on pattern recognition software 

\end{itemize} 

\noindent\hfil\rule{\textwidth}{.4pt}\hfil

\subsubsection*{Accomplishments}
We started by printing new circles without overlapping centers. We made sure to keep the radii the same so vision can find a radius that works for both circles. We then continued to work on the pattern recognition of the circles by testing different values of the minimum distance between the centers of the circles. We took a picture of the circles and use it to test the different values. We added extra parameters to reduce the number of circles found in the background. These parameters include minimum radius and maximum radius. This allows us to classify the correct circles more accurately. We draw the outlines of the two circles, each with a different color to differentiate between the two circles and the ones found in the background. We continued to try different values for the minimum distance based on the distance from the webcam the circles would be placed. 

\whatsnext{
\begin{itemize}
    \item Use sliders in Eclipse to test different values
\end{itemize} 
}

