%%% PACKAGES 
%%%----------------------------------------------------
\usepackage[utf8]{inputenc} % If utf8 encoding
\usepackage[T1]{fontenc}    %
\usepackage[english]{babel} % English please
\usepackage{lipsum} % generates fake test (testing)
%\usepackage[final]{microtype} % Less badboxes
\usepackage{blindtext}
%\usepackage{showframe}
%\usepackage{kpfonts} %Font
\usepackage{amsmath,amssymb,mathtools} % Math
%\usepackage[table,xcdraw]{xcolor} %for chart
\usepackage{tikz} % Figures
\usepackage{multicol}
%\usetikzlibrary{trees}
\usepackage{graphicx} % Include figures  NOTE: include [demo] to see black boxes
%\setkeys{Gin}{quiet}
%\setkeys{Gin}{keepaspectratio,width=2in,height=2in}
%\usepackage[margin=10pt,font=small,labelfont=bf,justification=raggedright]{caption}
%\usepackage{subcaption}  // memior already has it
%\usepackage{subfigure}
\usepackage{pgfplots}
\pgfplotsset{compat=1.14}
%\pgfplotsset{}
\usepackage{wrapfig}
\usepackage{soul}
\usepackage{pagenote}
\usepackage{array}
\usepackage{listings}
%\usepackage[export]{adjustbox}
%\usepackage[breaklinks]{hyperref}
\usepackage{placeins}
\usepackage[tikz]{bclogo}
\usepackage{algorithm}
\usepackage{algorithmic}

\usepackage{fancyvrb}

\usepackage{tikz}
\usetikzlibrary{shapes,arrows,positioning}
\usepackage{geometry}
    \geometry{
    left=0.15\paperwidth,
    right=0.15\paperwidth,
    top=0.1\paperwidth,
    bottom=0.1\paperwidth
    }

\usepackage{titlesec}

%\usepackage{hyperref} %breaks everything.  Need to look into it.
\def\hhsurl#1{\expandafter\string\csname #1\endcsname}

%Get rid of anoying pdf message about version 1.5
%\pdfminorversion=6

%The "booktabs" package makes nice tables. Sparing use of lines makes tables look much nicer (less "grid" like). %Multirow lets you make elements that span multiple rows or columns. The the package docs for examples.
%\usepackage{booktabs}
%\usepackage{multirow}

\errorcontextlines 10000
% Use \raggedbottom to stop latex from trying to add space 
% between sections to try to fill the page with text up to the 
% bottom
\raggedbottom

% \makeatletter
% % define a user command to choose the image
% % this command also creates an internal command to insert the image
% \def\@partimage{}
% \newcommand{\partimage}[2][]{\gdef\@partimage{\includegraphics[#1]{#2}}}
% \renewcommand{\printparttitle}[1]{\parttitlefont #1\vfil\@partimage\vfil}
% \makeatother

%%% PAGE LAYOUT 
%%%----------------------------------------------------

%\setlrmarginsandblock{0.15\paperwidth}{*}{1} % Left and right margin
%\setulmarginsandblock{0.1\paperwidth}{*}{1}  % Upper and lower margin
%\checkandfixthelayout

%\setbinding{4mm}
%\settypeblocksize{*}{24.5pc}{1.618}
%\setlrmargins{*}{*}{1.5}
%\setulmarginsandblock{6pc}{7pc}{*}
%\setmarginnotes{6pt}{6pc}{12pt}
%\strictpagecheck
%\setlength{\topskip}{1.6\topskip}
%\checkandfixthelayout

%%% SECTIONAL DIVISIONS
%%%----------------------------------------------------

%%%%%%%%%%%%%%%%%%%%%%%%%%%%%%%%%%%%%%%%%%%%%%%
% reset chapter numbers on each part
%
\makeatletter
\@addtoreset{chapter}{part}
\makeatother

%%%%%%%%%%%%%%%%%%%%%%%%%%%%%%%%%%%%%%%%%%%%%%%
% Part: center title
%
\titleformat{\part}[display]
{\normalfont\LARGE\color{red}\bfseries\centering}{}{0pt}{}
[{\titlerule[0.8pt]}]

%%%%%%%%%%%%%%%%%%%%%%%%%%%%%%%%%%%%%%%%%%%%%%%
% SubSection: center title
%
\titleformat{\subsection}[display]
{\normalfont\LARGE\color{blue}\bfseries\centering}{}{0pt}{}

%%%%%%%%%%%%%%%%%%%%%%%%%%%%%%%%%%%%%%%%%%%%%%%
% Remove section headers
%
\makeatletter
\titleformat{\section}[runin]{}{}{0pt}{\@gobble}
%\titleformat{\subsection}[runin]{}{}{0pt}{\@gobble}
\makeatother
%\titlespacing{\section}{\parindent}{0pt}{0pt}
\titlespacing{\subsection}{\parindent}{0pt}{0pt}


%\maxsecnumdepth{subsection} % Subsections (and higher) are numbered
%\setsecnumdepth{subsection}


%\setsecheadstyle{\normalfont\large\bfseries}
%\setsubsecheadstyle{\normalfont\normalsize\bfseries}
%\setparaheadstyle{\normalfont\normalsize\bfseries}
%\setparaindent{0pt}\setafterparaskip{0pt}

%%% FLOATS AND CAPTIONS
%%%----------------------------------------------------

\makeatletter                  % You do not need to write [htpb] all the time
\renewcommand\fps@figure{htbp} %
\renewcommand\fps@table{htbp}  %
\makeatother                   %

%\captiondelim{\null\newline}
%\captionnamefont{\small\sffamily\bfseries}
%\captiontitlefont{\small\sffamily\linespread{.84}\selectfont}
%\captionstyle[\centering]{\raggedright}
%\normalcaptionwidth
%\captiontitlefinal{.}
%\setlength\belowcaptionskip{.5ex}

% Don: breaks side by side figures
%\captiondelim{\space } % A space between caption name and text
%\captionnamefont{\small\bfseries} % Font of the caption name
%\captiontitlefont{\small\normalfont} % Font of the caption text

%\changecaptionwidth          % Change the width of the caption
%\captionwidth{1\textwidth} %

%%% ABSTRACT
%%%----------------------------------------------------

%\renewcommand{\abstractnamefont}{\normalfont\small\bfseries} % Font of abstract title
%\setlength{\absleftindent}{0.1\textwidth} % Width of abstract
%\setlength{\absrightindent}{\absleftindent}

%%% HEADER AND FOOTER 
%%%----------------------------------------------------


%
% Add graphics directories to search
%
%\graphicspath{{Images/}}

%%% TABLE OF CONTENTS
%%%----------------------------------------------------

% Only parts, chapters and sections in the table of contents
%\maxtocdepth{subsection} 
%\settocdepth{subsection}

% Make overflow at end of line for big pages numbers go away
%\setpnumwidth{2.5em}

% Center part number on line and add color
%\makeatletter
%\renewcommand{\partnumberline}[1]{\sffamily\large\bfseries\color{blue}\hfil\hspace\@tocrmarg Part #1:~}
%\makeatother
%\renewcommand{\cftpartfont}{\large\normalfont}
%\cftpagenumbersoff{part}

% Add Colored text before each "part" of TOC
%\cftpagenumbersoff{part}
%\renewcommand{\cftpartpresnum}{\sffamily\large\bfseries\color{blue}\partname\ }
%\renewcommand{\cftpartaftersnumb}{\\}
%\cftsetindents{part}{0em}{0em}

% Center Part Numbers in TOC
%\makeatletter
%\renewcommand\partnumberline[1]{\hfil\hspace\@tocrmarg #1~}
%\makeatother

% Add a \par to the end of the TOC
\AtEndDocument{\addtocontents{toc}{\par}} 

% Add "Meeting" at the start of each TOC entry
%\renewcommand{\cftchaptername}{Meeting\space}

%%% INTERNAL HYPERLINKS
%%%----------------------------------------------------

%\usepackage{hyperref}   % Internal hyperlinks
%\hypersetup{
%pdfborder={0 0 0},      % No borders around internal hyperlinks
%pdfauthor={I am the Author} % author
%}
%\usepackage{memhfixc}

%%% NEW COMMANDS
%%%----------------------------------------------------
\usepackage{wrapfig}
\usepackage{calc}
\usepackage{adjustbox}
\usepackage{graphicx}
\usepackage{float}

\newcommand{\insertbio}[7]{
	\newpage	
	\pagestyle{hhsstyle}
	\begin{flushright}
	\textbf{\Huge #1}
	\end{flushright}
	%\addcontentsline{toc}{section}{\numberline{}#1}
	\section{#1}
    %\index{#1} 
	%\newlength{\strutheight}
	%\settoheight{\strutheight}{\strut}
	\begin{adjustbox}{valign=T,
    				%raise=\strutheight,
                    minipage={\linewidth}}
    	\begin{wrapfigure}{R}{0pt}
    		% Portrait shot: looks best in 2:3 ratio, square is acceptable
            \includegraphics[width=6cm,height=5cm,keepaspectratio, angle = 270]{Bios/Images/#2} 
            %\includegraphics[width=6cm,height=5cm,angle = 270]{Bios/Images/#2} 
    	\end{wrapfigure}
    	\strut{}
    \end{adjustbox}
    \subsection*{Quick Facts}
	\begin{itemize}
		\item \textsc{\large Interests:} #3
		\item \textsc{\large Sub-Team:} #4
		\item \textsc{\large Year:} #5
	\end{itemize}
	\vspace{1cm}		
	\subsection*{Biography}
	#6	
	\begin{figure}[H]
	\centering
	\begin{minipage}{.5\textwidth}
	  \centering
	  \includegraphics[width=0.95\linewidth]{Bios/Images/#7}
	  %\captionof{figure}{Figure 1}
	  %\label{fig:test1}
	\end{minipage}%
	\end{figure}
}



\usepackage{wrapfig}
\usepackage{calc}
\usepackage{adjustbox}
\usepackage{graphicx}
\usepackage{float}

\newcommand{\insertbiomentor}[6]{
	%\newpage	
	\section{#1}
	%\pagestyle{plain}
	\begin{flushright}
	\textbf{\Huge #1}
	\end{flushright}
	%\addcontentsline{toc}{section}{\numberline{}#1}
    %\index{#1} 
	%\newlength{\strutheight}
	%\settoheight{\strutheight}{\strut}
	\begin{adjustbox}{valign=T,
    				%raise=\strutheight,
                    minipage={\linewidth}}
    	\begin{wrapfigure}{R}{0pt}
    		%\includegraphics[width=160px]{Bios/#2} 
            \includegraphics[width=7cm,height=7cm,keepaspectratio]{Bios/Images/#2} 
            % Portrait shot: looks best in 2:3 ratio, square is acceptable
    	\end{wrapfigure}
    	\strut{}
    \end{adjustbox}
    \subsection*{Quick Facts}
	\begin{itemize}
		\item \textsc{\large Occupation:} #3
		\item \textsc{\large Relation to Team:} #4
        \item \textsc{\large Interests:} #5
	\end{itemize}
	\vspace{3cm}		
	\subsection*{Biography}
	#6	
}

\newcommand{\insertCV}[7]{

\flushleft

\begin{quotebox}
\textsl{\Huge #2} \\
\flushright{\textrm{\large{\textbf{- #3}}}}
\end{quotebox}

\raggedright
\vskip 0.2in
\textbf{\huge #1}
\vskip 0.1in

\begin{figure}[h!]
\centering
\begin{minipage}{.48\textwidth}
  \centering
  \includegraphics[width= .8\linewidth]{Core_Values/#6}
\end{minipage}%
\hfill
\begin{minipage}{.48\textwidth}
  \centering
  \includegraphics[width= .8\linewidth]{Core_Values/#7}
\end{minipage}
\end{figure}

\begin{centering}
#4
\end{centering}
\vskip 0.2in
Refer to these pages to see more of this Core Value:
\vskip 0.2in
\begin{itemize}
#5 %use \item[$\blacksquare$]
\end{itemize}
}
\usepackage{wrapfig}
\usepackage{calc}
\usepackage{adjustbox}
\usepackage{graphicx}
\usepackage{float}

\newcommand{\insertOutreachEvent}[8]{
	\newpage	
	\pagestyle{plain}
	{
	\setthemecolor[DmgLilac]
	\begin{commentbox}{\flushleft{\huge{{\textrm{#1}}}}}
\flushleft
\textrm{\textbf{Date:} #2} \\
\textrm{\textbf{Event Total:} #3 Hours} \\
	\end{commentbox}
	}
	
	\addcontentsline{toc}{section}{\numberline{}#1}

	{
    \noindent\makebox[\textwidth][c]{%
	\begin{minipage}{\textwidth}
	  \centering
	  \includegraphics[width=0.75\linewidth]{Outreach/Images/#4}
	  %\captionof{figure}{Figure 1}
	  %\label{fig:test1}
	\end{minipage}}%
    }
    
    \setthemecolor[DmgLavender]	
    \begin{paperbox}{\flushleft{{Purpose:}}}
    
    \flushleft{#5}
    
    \end{paperbox}
    
    {\flushleft{\textbf{Description:}}} \\*

	#6	
	% \begin{figure}[H]
	% \centering
	% \begin{minipage}{.5\textwidth}
	%   \centering
	%   \includegraphics[width=0.9\linewidth]{Outreach/Images/#7}
	%   %\captionof{figure}{Figure 1}
	%   %\label{fig:test1}
	% \end{minipage}%
	% \begin{minipage}{.5\textwidth}
	%   \centering
	%   \includegraphics[width=0.9\linewidth]{Outreach/Images/#8}
	%   %\captionof{figure}{Figure 2}
	%   %\label{fig:test2}
	% \end{minipage}
	% \end{figure}

 \begin{figure}[h!]
 \centering
   \includegraphics[width=\textwidth, angle=0]{Outreach/Images/#7}
 \end{figure}

 \begin{figure}[h!]
 \centering
   \includegraphics[width=\textwidth, angle=0]{Outreach/Images/#8}
 \end{figure}
}
%%%%%%%%%%%%%%%%%%%%%%%%%%%%%%%%%%%%%%%%%%%%%%
%             insertCompetition
% 1) Competition Name (ex. Ruckus at the Rock)
% 2) Date (MM/DD/YYYY)
% 3) Location (ex. Hagerty High School)
% 4) Meet Ranking (ex. 5th)
% 5) Match Results Breakdown (image file path)
% 6) Match Scoring Pie Chart (image file path)
% 7) Competition OPR Breakdown (image file path)
% 8) Reflection (text)
%%%%%%%%%%%%%%%%%%%%%%%%%%%%%%%%%%%%%%%%%%%%%%

\usepackage{wrapfig}
\usepackage{calc}
\usepackage{adjustbox}
\usepackage{graphicx}
\usepackage{float}

\newcommand{\insertCompetition}[8]{
	\newpage	
	\pagestyle{plain}
	%\begin{flushleft}
    \begin{description}
		\item [\Huge Competition:] {\Huge #1}
		\item [Date:] #2
		\item [Location:] #3
        \item [Meet Ranking: #4]
	\end{description} 
	\addcontentsline{toc}{section}{\numberline C#1 Competition: #2}

	\begin{figure}[h!]
    \centering
    \includegraphics[width=0.8\textwidth, angle=0]{Competition/Images/#5}
    \end{figure}
    
    \begin{figure}[h!]
	\centering
	\begin{minipage}{.5\textwidth}
	  \centering
	  \includegraphics[width=0.95\linewidth]{Competition/Images/#6}
	\end{minipage}%
	\begin{minipage}{.5\textwidth}
	  \centering
	  \includegraphics[width=0.95\linewidth]{Competition/Images/#7}
	\end{minipage}
	\end{figure}
    
	\subsection*{What happened:}
    {#8}
    
 }

%%%%%%%%%%%%%%%%%%%%%%%%%%%%%%%%%%%%%%%%%%%%%%
%                insertmeeting
% 1) Title (something creative & funny?)
% 2) Date (MM/DD/YYYY)
% 3) Location (ex. Hagerty High School)
% 4) People/Committees Present 
% 5) Picture 
% 6) Start Time & Stop Time (ex. 12:30AM to 4:30PM)
%%%%%%%%%%%%%%%%%%%%%%%%%%%%%%%%%%%%%%%%%%%%%%

\newcommand{\insertmeeting}[7]{
% pagestyle
%   meeting - chapter and section on the same heading
%   headings - 
%   ruled - horizontal line in the heading
\section*{#1}
\pagestyle{hhsstyle}
\addcontentsline{toc}{section}{\numberline{}#2 - #1}
% \section[#2 - #1]{#2}
% \chapter[#2 - #1]{#2}
% \index{#2} 

\begin{figure}[h]
\begin{minipage}[b]{0.6\linewidth}
    \begin{description}
        \item [\Large \textcolor{black}{#1}]
        \item #2
        \item #3
        \item #4
        \item #6
    \end{description} 
\end{minipage}%
\hfill
\begin{minipage}[b]{0.35\linewidth}
	\raggedleft

    %\includegraphics[scale=0.3]{#5}
    %\setlength{\fboxsep}{0pt}\fbox{
    %\includegraphics[width=.3\linewidth]
    %\frame{\includegraphics[width=3cm,height=3cm,keepaspectratio]
    \fbox{\includegraphics[width=3cm,height=3cm,keepaspectratio]
    {#5}}
    \centering
    \par Robot
\end{minipage}
%\noindent\rule{\textwidth}{1pt}
\vspace{5mm}
\begingroup % because of color
	\color{gray}%
	\hrule width \hsize \kern 1mm \hrule width \hsize height 2pt 
\endgroup

\end{figure}%
}  %end newcommand

%%%%%%%%%%%%%%%%%%%%%%%%%%%%%%%%%%%%%%%%%%%%%%
%                newsection
%
% 1) Committee/Subcommittee Name (ex. Hardware & Design)
% 2) List of Goals (remember to use \item)
% 3) Accomplishments (written in third person collective)
%%%%%%%%%%%%%%%%%%%%%%%%%%%%%%%%%%%%%%%%%%%%%%

\newcommand{\newsection}[4]{
\addcontentsline{toc}{subsection}{\numberline{}#1}

\subsection*{#1}
\noindent\hfil\rule{\textwidth}{.4pt}\hfil
\begin{multicols}{2}

\setthemecolor[PhbLightGreen]

\begin{paperbox}{\flushleft{Goals:}}

{
\flushleft
\begin{itemize}
#2
\end{itemize}
}

\end{paperbox}

{\flushleft{\textbf{\large ACCOMPLISHMENTS:}}} \\*
\vspace{2mm}
#3

\end{multicols}


{

\setthemecolor[DmgLilac]
\begin{paperbox}[float=!h]{\flushleft{Next on the Docket:}}
\flushleft{#4}
\end{paperbox}

}

}  %end newcommand

%%%%%%%%%%%%%%%%%%%%%%%%%%%%%%%%%%%%%%%%%%%%%%
%                insertimage
%
% 1) Image File Path 
% 2) Caption (Camel Case, Short & Descriptive)
% 3) Label (UNIQUE identifier for each image)
%%%%%%%%%%%%%%%%%%%%%%%%%%%%%%%%%%%%%%%%%%%%%%

\newcommand{\insertmeetingimage}[3]{

\begin{wrapfigure}{c}{0.75\linewidth}
\includegraphics[width=\linewidth]{#1}
\caption{#2}
\label{fig:#3}
\end{wrapfigure}

}
%This allows the meeting chapters to show up on the toc but not have a header page :)

\makeatletter
\newcommand{\unchapter}[1]{%
  \begingroup
  \let\@makechapterhead\@gobble % make \@makechapterhead do nothing
  \chapter{#1}
  \endgroup
}
\makeatother
\usepackage{wrapfig}
\usepackage{calc}
\usepackage{adjustbox}
\usepackage{graphicx}
\usepackage{float}

\newcommand{\insertdesignoverview}[6]{
	\newpage	
	\pagestyle{plain}
	\begin{flushright}
	\textbf{\Huge #1}
	\end{flushright}
	\addcontentsline{toc}{section}{\numberline{}#1}
    %\index{#1} 
	%\newlength{\strutheight}
	%\settoheight{\strutheight}{\strut}
   
 %\vspace{.25cm}		
	\subsection*{Goal: \hl{#2}}

   \begin{figure}[H]
	\centering
	\begin{minipage}{.5\textwidth}
	  \centering
	  \includegraphics[width=0.95\linewidth]{Design_Overview/#3}
	  %\captionof{figure}{Figure 1}
	  %\label{fig:test1}
	\end{minipage}%
	\begin{minipage}{.5\textwidth}
	  \centering
	  \includegraphics[width=0.95\linewidth]{Design_Overview/#4}
	  %\captionof{figure}{Figure 2}
	  %\label{fig:test2}
	\end{minipage}
	\end{figure}
    
%	\vspace{.25cm}		
	\subsection*{Core Materials}
	#5

%	\vspace{.25cm}		
	\subsection*{Manufacturing Processes}
	#6
	\vspace{.1cm}


	
%	\vspace{.25cm}		
%	\subsection*{Description}

}
\usepackage{adjustbox}

%%%%%%%%%%%%%%%%%%%%%%%%%%%%%%%%%%%%%%%%%%%%%%%%%%
% 1 Name
% 2 Bio Image
% 3 Quote
% 4 Committees
% 5 Current grade
% 6 Leave Blank
% 7 Stats, with & between each stat
%%%%%%%%%%%%%%%%%%%%%%%%%%%%%%%%%%%%%%%%%%%%%%%%%%
\newcommand{\dndbio}[7]{
	\newpage
	\addcontentsline{toc}{section}{\numberline{}#1}
	\begin{figure}
	\centering
	\includegraphics[height=.45\paperwidth, keepaspectratio]{Bios/Images/#2}
	\end{figure}
	\begin{monsterboxnobg}{#1}
		\begin{hangingpar}
	    	\textit{``#3''}
	  	\end{hangingpar}
	\dndline
	\basics[%
		committees = \Large{#4},
		Current Grade = \Large{#5},
		Stats = \Large{#6}
	]
	\vspace{3mm}
	\stats[
	    STAT = \stat{#7}
	]
	\dndline%
	\end{monsterboxnobg}
}


%%%%%%%%%%%%%%%%%%%%%%%%%%%%%%%%%%%%%%%%%%%%%%%%%%
% Use \monstersection{"Title"} beforehand to create a section (which is just a list), then use this command to create entries in the list
% 1 Entry
% 2 Description
%%%%%%%%%%%%%%%%%%%%%%%%%%%%%%%%%%%%%%%%%%%%%%%%%%
\newcommand{\dndaction}[2]{
	\begin{monsteraction}[\large{#1}]
	        #2
	    \end{monsteraction}
}
\usepackage{wrapfig}


%%%%%%%%%%%%%%%%%%%%%%%%%%%%%%%%%
% 1 5-6 line introduction of current finances
% 2 Funding goals image path
% 3 Label for the funding goals
% 4 Table of funds 
% 5 Label for table
% 6 Pie Chart of breakdown
% 7 Label for pie chart
% 8 Bar chart of how we use our money
% 9 Label for the use
%%%%%%%%%%%%%%%%%%%%%%%%%%%%%%%%%
\newcommand{\financemodule}[9]{
	
	\newpage
	\addcontentsline{toc}{section}{Business Plan}
	\large{\textbf{Current Financials: }#1}\\

	% make this a lot longer, fill 4-5 lines in the pdf
	\large{Our original funding goals can be visualized through \ref{fig:#3}.}\\

	\begin{figure}[h!]
    	\centering
    	\includegraphics[width=0.6\textwidth]{#2}
		\label{fig:#3}
		\caption{Current Funding Goals}	
	\end{figure}

	\large{We receive funds through multiple sources, with \ref{fig:#5} showing the complete breakdown of our team's funding, and the source of funding is shown through \ref{fig:#7}.}\\

	\newpage
	\begin{figure}[h!]
		\begin{minipage}{.5\textwidth}
			\centering
	    	\includegraphics[width=0.98\linewidth]{#4}
			\label{fig:#5}
			\caption{The team's total funding}
		\end{minipage}%
		\begin{minipage}{.5\textwidth}
			\centering
	    	\includegraphics[width=0.98\linewidth]{#6}
			\label{fig:#7}
			\caption{A breakdown of the funding}
	\end{minipage}
	\end{figure}

	\begin{figure}[h!]
		\centering
		\includegraphics[width=.6\textwidth]{#8}
		\label{fig:#9}
		\caption{How we use our money}
	\end{figure}

	\Large{"Some comparison of how we expected to spend our money and how we actually spended our money"}
}

%%%%%%%%%%%%%%%%%%%%%%%%%%%%%%%%%%%%%%%%%%%%%%
% new itemize enviroment that doesn't allow latex to put 
% more space between items when it tries to fill the entire 
% page with text.  If you use /raggedbottom then this is not needed.
\newenvironment{myitemize}
{ \begin{itemize}
    \setlength{\itemsep}{0pt}
    \setlength{\parskip}{0pt}
    \setlength{\parsep}{0pt}     }
{ \end{itemize}  }


%%%%%%%%%%%%%%%%%%%%%%%%%%%%%%%%%%%%%%%%%%%%%%
% Pagenote section
% Use page notes to mark control and innovation
\makepagenote
%\continuousnotenums
%\notepageref
\renewcommand*{\pagenotesubhead}[3]{}

%%%%%%%%%%%%%%%%%%%%%%%%%%%%%%%%%%%%%%%%%%%%%%
% Control label
% 
\makeatletter
\newcommand{\interesting}[2]{%
  \@bsphack
  \csname phantomsection\endcsname % in case hyperref is used
  \def\@currentlabel{#1}{\label{#2}}%
  \@esphack
}
\makeatother

%%%%%%%%%%%%%%%%%%%%%%%%%%%%%%%%%%%%%%%%%%%%%%
% Interesting Page Table
% 
\newcommand{\interestingpagetable}{
	\newpage	
	\pagestyle{plain}
	\textbf{\Huge Interesting Pages}
	\addcontentsline{toc}{section}{\numberline{}Interesting Pages}
  
 	\vspace*{1cm}
	 %\par 
    We have created the table below that highlights some of the more interesting pages in our notebook.  The highlighted ones are also tabbed.
    
    \vspace*{1cm}
    
    % more space between rows
    \renewcommand{\arraystretch}{1.2} 
    % Remove space to the vertical edges
    %\begin{tabular}{@{}lll@{}}
    
    %\interesting{This is an example of a good control}{control:1} % enter this in the actual page file
    
\resizebox{\textwidth}{!}{%
    \begin{tabular}{lp{0.7\textwidth}r}
		\toprule
		Type & Description & Page \\ 
	\midrule
            
            Innovate & \ref{innovate:1} & \pageref{innovate:1}\\
            Innovate & \ref{innovate:3} & \pageref{innovate:3}\\
            Innovate & \ref{Innovate:4} & \pageref{Innovate:4}\\
            Innovate & \ref{Innovate:55} & \pageref{Innovate:55}\\
            Design & \ref{design:1} & \pageref{design:1}\\
            Design & \ref{design:2} & \pageref{design:2}\\
            Design & \ref{design:3} & \pageref{design:3}\\
            Design & \ref{design:5} & \pageref{design:5}\\
            Design & \ref{design:6} & \pageref{design:6}\\
            Control & \ref{control:1} & \pageref{control:1}\\
            Control & \ref{control:2} & \pageref{control:2}\\
            Control & \ref{control:3} & \pageref{control:3}\\
            Control & \ref{control:4} & \pageref{control:4}\\
            Control & \ref{innovate:2} & \pageref{innovate:2}\\
            Control & \ref{control:77} & \pageref{control:77}\\
            Control & \ref{control:44} & \pageref{control:44}\\
            Control & \ref{control:99} & \pageref{control:99}\\
            Connect & \ref{connect:1} & \pageref{connect:1}\\
            Connect & \ref{connect:2} & \pageref{connect:2}\\
            Connect & \ref{connect:3} & \pageref{connect:3}\\
            Connect & \ref{connect:4} & \pageref{connect:4}\\
            Think & \ref{timeline:1} & \pageref{timeline:1}\\
            Think & \ref{think:1} & \pageref{think:1}\\
            Think & \ref{think:2} & \pageref{think:2}\\
            Competition & \ref{competition:1} & \pageref{competition:1}\\
            Competition & \ref{competition:2} & \pageref{competition:2}\\
            Competition & \ref{competition:3} & \pageref{competition:3}\\
	
           
		\bottomrule
	\end{tabular}
}
    \vspace{.5cm}
\newline \textbf{Recent Update: We are now balancing without touching the ground with our arms!} For more information, please see the control sections.
}

\makepagenote

% %Code to make Figures stay within their \section (i.e. Programming):
% \makeatletter
%     \renewcommand\section{\@startsection {section}{1}{\z@}%
%        {-3.5ex \@plus -1ex \@minus -.2ex}%
%        {2.3ex \@plus.2ex}%
%        {\FloatBarrier\normalfont\large\bfseries}}
% \makeatother

%Code to make Figures stay within their \subsection (i.e. Accomplishments):
\makeatletter
\AtBeginDocument{%
  \expandafter\renewcommand\expandafter\subsection\expandafter
    {\expandafter\@fb@secFB\subsection}%
  \newcommand\@fb@secFB{\FloatBarrier
    \gdef\@fb@afterHHook{\@fb@topbarrier \gdef\@fb@afterHHook{}}}%
  \g@addto@macro\@afterheading{\@fb@afterHHook}%
  \gdef\@fb@afterHHook{}%
}
\makeatother

\definecolor{pblue}{rgb}{0.13,0.13,1}
\definecolor{pgreen}{rgb}{0,0.5,0}
\definecolor{pred}{rgb}{0.9,0,0}
\definecolor{pgrey}{rgb}{0.46,0.45,0.48}

\lstset{language=Java,
  showspaces=false,
  showtabs=false,
  breaklines=true,
  showstringspaces=false,
  breakatwhitespace=true,
  commentstyle=\color{pgreen},
  keywordstyle=\color{pblue},
  stringstyle=\color{pred},
  basicstyle=\ttfamily,
  moredelim=[il][\textcolor{pgrey}]{$$},
  moredelim=[is][\textcolor{pgrey}]{\%\%}{\%\%}
}
