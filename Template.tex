%/Template RN
% Fill in any thing that is in quotes. ("") 
% % What everything in lines 11 - 16 are. 
% % 1) Title
% % 2) Date
% % 3) Location
% % 4) Those Present
% % 5) Picture(s)
% % 6) Weight(almost always 1lb)

\insertmeeting 
	{"Title"} 
	{"MM/DD"/YYYY}
	{"Location"}
	{"People Who Were Present"}
	{RobotPics/robot5.jpg}%This calls on a picture from the RobotPics folder to be used as a thumbnail at the top right of the page, 
	{1lb}
	

%\hl{this is some highlighted text}		highlight text
%\textbf{}		bold face text


% Usually, all meetings start with a list of goals for the meeting. 
\noindent\hfil\rule{\textwidth}{.4pt}\hfil
\section{"Section Name"}
\subsection*{Goals}
\begin{itemize}
	\item "Write Goals" 
\end{itemize} 

% Below Creates a line across the paper. 
\noindent\hfil\rule{\textwidth}{.4pt}\hfil

% Ignore this. 
% \noindent\hfil\rule{\textwidth}{.4pt}\hfil
% \subsection{People Working On It}
% \begin{itemize}
% 	\item "Write Names" 
% \end{itemize} 

\noindent\hfil\rule{\textwidth}{.4pt}\hfil
\subsection*{"Results"}
	"Write results here." 


% Picture Insert Code
%  \begin{figure}[h!]
%  \centering
%    \includegraphics[width=0.4\textwidth, angle=0]{Meeting/January/01-10-17/Cap_Ball_Lifter_Drawing.png}
%   \caption{Hand drawing of claw grabber and cap ball mechanisms}
%   \label{fig:Cap_Ball_Lifter_Drawing}
%  \end{figure}

% Picture Positioning:
% h - puts the image exactly where you want it in the document
% ! - tells latex to ignore some of its other rules to make your thing happen
% t - try to put it at the top of the page
% b - try to put it at the bottom of the page
% p - make a special page for the float
% H - Really really put it here

% Example Table 
% A B C
% D E F
% G H I 
    \begin{tabular}{lp{0.7\textwidth}r}
		\toprule
		A & B & C \\
		\midrule
        D & E & F \\ 
        G & H & I \\ 
		\bottomrule
	\end{tabular}
    
    
    
    
% Creating Double Picture Minipages
% \begin{figure}[htp]
% \centering
% \begin{minipage}{.48\textwidth}
%   \centering
%   \includegraphics[width=\linewidth]{Meetings/February/02-08-18/GlyphDitchIssue.jpg}
%   \caption{Glyph Getting Caught Underneath the Back Plate}
%   \label{fig:front}
% \end{minipage}%
% \hfill
% \begin{minipage}{.48\textwidth}
%   \centering
%   \includegraphics[width=\linewidth]{Meetings/February/02-08-18/GlyphDitchIssue.jpg}
%   \caption{Glyph Getting Caught Underneath Kicker}
%   \label{fig:back}
% \end{minipage}
% \end{figure}


%This is how you reference a figure in text.
%As shown in Figure \ref{fig:jewel}


Horizontal spaces between figures are controlled by one of several commands, which are placed in between \begin{subfigure} and \end{subfigure}:

A non-breaking space (specified by ~ as in the example above) can be used to insert a space in between the subfigs.
Math spaces: \qquad, \quad, \;, and \,
Generic space: \hspace{''length''}
Automatically expanding/contracting space: \hfill

% Fix overflows in tables
\resizebox{\textwidth}{!}{%
 
