\addcontentsline{toc}{chapter}{\numberline{}Lessons Learned}
\subsection*{\textbf{\Huge Lessons Learned:}}
\vspace{.2cm}
%\begin{flushleft}
\setlength{\parindent}{.25in} 
\interesting{   }{LessonsLearned:1}
Every season, we as the Mechromancers regard our competitions as being \hl{less about winning or losing}, and more as a learning experience, with \hl{every success and failure acting as a lesson}, providing the bright minds on our team with real engineering insight into what works and what doesn't. The Mechromancers realize the significance of the process of trial and error in the engineering and design process, and understand how much it contributes to \hl{encouraging creative thinking and innovation within the team}. For this reason, we have always recorded our results in meets and competitions, and present what we learned on competition day in a small, bulleted section called "lessons learned" within each competition entry. The Relic Recovery season featured an abundance of adaptation and change through experience for the Mechromancers. Here, you can learn about the \hl{difficulties we faced as a team within each competition, how we overcame these difficulties through design changes and new innovations, and the reasoning behind each new iteration of Buzz} that the Mechromancers conjured over the season. 
\begin{figure}
 \centering
 \begin{minipage}{.48\textwidth}
   \centering
   \includegraphics[width=\linewidth]{Competition/Old.JPG}
   \caption{The Mechromancers' First Iteration of Buzz}
   \label{fig:FirstIt}
 \end{minipage}%
\hfill
 \begin{minipage}{.48\textwidth}
   \centering
   \includegraphics[width=\linewidth]{Competition/New.PNG}
   \caption{Buzz after the Season's Evolutions}
   \label{fig:Evolved}
 \end{minipage}
 \end{figure}
The first permutation of Buzz holds almost no resemblance to the Buzz that the world knows today, and only goes to show how the team adapts to challenges and adversity. Before our very first meet of the season on November 11th, the Mechromancers planned to design a robot with \hl{simplicity, speed and modularity in mind} that could easily pick up glyphs and score them within the cryptobox, and could \hl{possibly be automated easily for multiple glyph delivery}. We designed a fast and modular drivetrain, with our intake mechanism attached to two sliding rails, supplementing an elevator system. This elevator, consisting of a 20:1 motor powering a series of pulleys that pulled the intake mechanism up and down, allowed a simple method of glyph delivery that featured the glyphs to be intaked and delivered using the same mechanism. Although simplistic in design, this permutation of Buzz featured several design flaws that inhibited efficient glyph delivery. For example, one issue that the team struggled with frequently shortly before our first competition was how \hl{the force of gravity counteracted the lifting of the elevator mechanism}, as for the elevator to remain at a set position, it required power to be supplied constantly and at a greater rate. At one point, this even led to our pulley motor browning out. We also noticed that the glyphs, once pulled into our intake, remained close together and made glyph delivery difficult, often causing glyphs to descore and fall out of the column. Another issue that we noticed with this iteration of Buzz was the \hl{lack of control} it offered. The robot's simplistic design inhibited the calculated manipulation of glyph delivery, or the lack of a chance to improve efficiency through autonomy. Recognizing the significance of these design flaws through our experiences leading up to and on competition day, the Mechromancers \hl{took these valuable lessons in consideration when designing a new glyph delivery mechanism that took advantage of the force of gravity in its glyph scoring, and featured more of an ability to control and manipulate the mechanism as well}, not only enabling better driver operated performance, but permitting further advancements in the autonomous period (such as our goal to score multiple glyphs in autonomous). 
  \begin{figure}[htp]
  \centering
    \includegraphics[width=\textwidth, angle=0]{Competition/Old2.JPG}
   \caption{Buzz at our First Meet on November 11th}
   \label{fig:Nov11}
  \end{figure}
The second version of Buzz, prevailing at our second meet of the season on December 9th, solved such design flaws magnificently. Not only did \hl{the arms take advantage of gravity, delivering glyphs at the back of the robot and allowing them to slide down} into the cryptobox columns, but they provided an increased ability to use sensors and algorithms to refine TeleOp and Autonomous performance. The sliders on \hl{the arms could be preprogrammed, the claws could be automated, and almost all aspects of glyph delivery could be done without driver control.} Best of all, it was a \hl{unique and innovative} method that was unparalleled to our competition. However, this didn't mean that we didn't face issues at the second meet. The glyph separation mechanism of using two belts to pull the glyphs apart malfunctioned frequently, and our glyph delivery was often slowed due to the precarious aligning required during delivery. Yet again, through what we'd learned on the competition field, we continued to innovate and make improvements, getting ready for League Championships. Now, the team began to \hl{fully utilize our sensors and algorithms to automate glyph delivery positions, the glyph pickup position, and the sychronization of both arms.} We also used a \hl{multitude of cameras to give Buzz vision, continuing to perfect the autonomous and working to complete the jewel hitter.} The amalgamation of the hardware of our robot with the software was what finally gave Buzz life. And although Buzz has already served the Mechromancers most of the entire season, the team agrees that the robot is nowhere close to finished, as the Mechromancers \hl{forever continue to innovate and improve.} 
  \begin{figure}[htp]
  \centering
    \includegraphics[width=\textwidth, angle=0]{Competition/New2.JPG}
   \caption{Buzz, New and Improved}
   \label{fig:NewBuzz}
  \end{figure}
  On the following pages, you'll find our competition entries throughout the season, with a description of what happened at each event as well as all of the lessons we learned through our performance on competition day. This is followed by our "lessons learned" subsection, as well as a match results table depicting our overall performance through our statistics on the field. 