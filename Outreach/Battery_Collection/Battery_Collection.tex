\insertoutreachHours{Battery Collection} 
{01/09/23}
{N/A}
{Battery_Collection/AllBatteryArrow.jpg}
{To participate in outreach related to this year's FIRST theme} 
{Our research on battery collection found us at a website titled Call2Recycle, an organization dedicated to the collection of batteries. It featured information on how to recycle batteries and where battery recycling locations are

Traditional battery collection can be separated into four types:

Rechargeable batteries: Very accessible to people in Seminole County since there are plenty of locations that accept them only 5 miles away, it is Federally mandated that collection programs for recyclables are free. While it is possible for us to do this, it wouldn't contribute a lot to the community due to how accessible it is.

Alkaline (Single use) batteries: The nearest location that collects these is an hour away, and there is no mandate that requires collection to be free, so this is likely our best bet in terms of making an impact.

Broken batteries: This requires a lot more preparation, namely to clean any toxic chemicals that leak as a result of the breakage. It would be difficult but it would still be possible, and there aren’t any nearby locations accept broken batteries either.

Cellular devices: Technically not a battery, but locations that recycle them are listed so they are likely similar to recycle. Nearby locations do exist though so it’s fairly moot.

In terms of how we can collect batteries, we would need to have a collection box. People would need to throw their batteries in here, properly concealing the ends to not cause any accidents, and we would need to get them sent to a location where they can be disposed of.

This does produce an issue however, it would be best to put this in a public location, and the only public location in the school is the front office. Thus, we would need permission from the school staff to place our collection box there. We would need to talk to the Admin in charge of extracurriculars

In terms of who we send batteries to, we had two options:
Call2Recycle, a battery collection organization. They work with 30,000 locations across the US and Canada, and they offer purchasable collection boxes. However, the collection boxes are expensive (cheapest is 45 dollars) and we couldn't find much information on if we have to send it to them, or if they will come and collect ours. If we have to send it to them it would get too expensive with shipping, and I'm not even sure if they will give back the box.

There was also Seminole County Waste Management, who is local enough to drive to and accepts batteries. If we had someone drive there to drop off batteries every 2 weeks to month, it would likely work much better. However, since we’re an organization, dropping off the batteries as one would cause an extra charge. This is likely the better option regardless as we know we won’t need to buy multiple boxes, it’s more local, and we don’t need to contact them beforehand.

Box building day
A few days back, we had created a poster to spread around our local campus, and an instructions sheet placed alongside the box to detail proper packaging. We brought in a few boxes, combined them together to make a larger one, and wrapped it in green paper, courtesy of our school’s media center. Once all of that was done, we set up the donation box in our school’s media center, and put up the posters around the school.
} 
{Battery_Collection/Web-Post-Logo.jpg}
{Battery_Collection/img.jpg}